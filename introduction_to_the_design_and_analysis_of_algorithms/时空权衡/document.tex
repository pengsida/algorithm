% !TeX spellcheck = en_US
%% 字体:方正静蕾简体
%%		 方正粗宋
\documentclass[a4paper,left=2.5cm,right=2.5cm,11pt]{article}

\usepackage[utf8]{inputenc}
\usepackage{fontspec}
\usepackage{cite}
\usepackage{xeCJK}
\usepackage{indentfirst}
\usepackage{titlesec}
\usepackage{longtable}
\usepackage{graphicx}
\usepackage{float}
\usepackage{rotating}
\usepackage{subfigure}
\usepackage{tabu}
\usepackage{amsmath}
\usepackage{setspace}
\usepackage{amsfonts}
\usepackage{appendix}
\usepackage{listings}
\usepackage{xcolor}
\usepackage{geometry}
\setcounter{secnumdepth}{4}
\usepackage{mhchem}
\usepackage{multirow}
\usepackage{extarrows}
\usepackage{hyperref}
\titleformat*{\section}{\LARGE}
\renewcommand\refname{参考文献}
\renewcommand{\abstractname}{\sihao \cjkfzcs 摘{  }要}
%\titleformat{\chapter}{\centering\bfseries\huge\wryh}{}{0.7em}{}{}
%\titleformat{\section}{\LARGE\bf}{\thesection}{1em}{}{}
\titleformat{\subsection}{\Large\bfseries}{\thesubsection}{1em}{}{}
\titleformat{\subsubsection}{\large\bfseries}{\thesubsubsection}{1em}{}{}
\renewcommand{\contentsname}{{\cjkfzcs \centerline{目{  } 录}}}
\setCJKfamilyfont{cjkhwxk}{STXingkai}
\setCJKfamilyfont{cjkfzcs}{STSongti-SC-Regular}
% \setCJKfamilyfont{cjkhwxk}{华文行楷}
% \setCJKfamilyfont{cjkfzcs}{方正粗宋简体}
\newcommand*{\cjkfzcs}{\CJKfamily{cjkfzcs}}
\newcommand*{\cjkhwxk}{\CJKfamily{cjkhwxk}}
\newfontfamily\wryh{Microsoft YaHei}
\newfontfamily\hwzs{STZhongsong}
\newfontfamily\hwst{STSong}
\newfontfamily\hwfs{STFangsong}
\newfontfamily\jljt{MicrosoftYaHei}
\newfontfamily\hwxk{STXingkai}
% \newfontfamily\hwzs{华文中宋}
% \newfontfamily\hwst{华文宋体}
% \newfontfamily\hwfs{华文仿宋}
% \newfontfamily\jljt{方正静蕾简体}
% \newfontfamily\hwxk{华文行楷}
\newcommand{\verylarge}{\fontsize{60pt}{\baselineskip}\selectfont}  
\newcommand{\chuhao}{\fontsize{44.9pt}{\baselineskip}\selectfont}  
\newcommand{\xiaochu}{\fontsize{38.5pt}{\baselineskip}\selectfont}  
\newcommand{\yihao}{\fontsize{27.8pt}{\baselineskip}\selectfont}  
\newcommand{\xiaoyi}{\fontsize{25.7pt}{\baselineskip}\selectfont}  
\newcommand{\erhao}{\fontsize{23.5pt}{\baselineskip}\selectfont}  
\newcommand{\xiaoerhao}{\fontsize{19.3pt}{\baselineskip}\selectfont} 
\newcommand{\sihao}{\fontsize{14pt}{\baselineskip}\selectfont}      % 字号设置  
\newcommand{\xiaosihao}{\fontsize{12pt}{\baselineskip}\selectfont}  % 字号设置  
\newcommand{\wuhao}{\fontsize{10.5pt}{\baselineskip}\selectfont}    % 字号设置  
\newcommand{\xiaowuhao}{\fontsize{9pt}{\baselineskip}\selectfont}   % 字号设置  
\newcommand{\liuhao}{\fontsize{7.875pt}{\baselineskip}\selectfont}  % 字号设置  
\newcommand{\qihao}{\fontsize{5.25pt}{\baselineskip}\selectfont}    % 字号设置 

\usepackage{diagbox}
\usepackage{multirow}
\boldmath
\XeTeXlinebreaklocale "zh"
\XeTeXlinebreakskip = 0pt plus 1pt minus 0.1pt
\definecolor{cred}{rgb}{0.8,0.8,0.8}
\definecolor{cgreen}{rgb}{0,0.3,0}
\definecolor{cpurple}{rgb}{0.5,0,0.35}
\definecolor{cdocblue}{rgb}{0,0,0.3}
\definecolor{cdark}{rgb}{0.95,1.0,1.0}
\lstset{
	language=C,
	numbers=left,
	numberstyle=\tiny\color{black},
	showspaces=false,
	showstringspaces=false,
	basicstyle=\scriptsize,
	keywordstyle=\color{purple},
	commentstyle=\itshape\color{cgreen},
	stringstyle=\color{blue},
	frame=lines,
	% escapeinside=``,
	extendedchars=true, 
	xleftmargin=1em,
	xrightmargin=1em, 
	backgroundcolor=\color{cred},
	aboveskip=1em,
	breaklines=true,
	tabsize=4
} 

\newfontfamily{\consolas}{Consolas}
\newfontfamily{\monaco}{Monaco}
\setmonofont[Mapping={}]{Consolas}	%英文引号之类的正常显示,相当于设置英文字体
\setsansfont{Consolas} %设置英文字体 Monaco, Consolas,  Fantasque Sans Mono
\setmainfont{Times New Roman}

\setCJKmainfont{华文中宋}


\newcommand{\fic}[1]{\begin{figure}[H]
		\center
		\includegraphics[width=0.8\textwidth]{#1}
	\end{figure}}
	
\newcommand{\sizedfic}[2]{\begin{figure}[H]
		\center
		\includegraphics[width=#1\textwidth]{#2}
	\end{figure}}

\newcommand{\codefile}[1]{\lstinputlisting{#1}}

% 改变段间隔
\setlength{\parskip}{0.2em}
\linespread{1.1}

\usepackage{lastpage}
\usepackage{fancyhdr}
\pagestyle{fancy}
\lhead{\space \qquad \space}
\chead{第七章 时空权衡 \qquad}
\rhead{\qquad\thepage/\pageref{LastPage}}
\begin{document}

\tableofcontents

\clearpage

\section{计数排序}
\subsection{比较计数}
\subsubsection{算法思路}
	针对待排序列表中的每一个元素,算出列表中小于该元素的元素个数,并把结果记录在一张表中,
	这个个数就指出了该元素在有序列表中的位置。

\subsubsection{算法实现代码}
	\begin{lstlisting}
	void comparison_counting_sort(int *A, int len)
	{
		int i,j;
		int* count = (int*)malloc(len * sizeof(int));
		int* result = (int*)malloc(len * sizeof(int));
		for(i = 0; i < len; ++i)
			count[i] = 0;
		for(i = 0; i < len; ++i)
		{
			for(j = i+1; j < len; ++j)
			{
				if(A[i] < A[j])
					++count[j];
				else
					++count[i];
			}
		}
		for(i = 0; i < len; ++i)
			result[count[i]] = A[i];
		for(i = 0; i < len; ++i)
			A[i] = result[i];
		free(count);
		free(result);
	}
	\end{lstlisting}

\subsubsection{算法分析}
	算法的基本操作为键值比较操作,基本操作的执行次数为
	\begin{equation}
		C(n) = \frac{(n-1)n}{2}
	\end{equation}

	算法的键值比较次数和选择排序一样多,并且还占用了线性数量的额外空间,所以不推荐它来做实际的应用。

\subsection{分布计数}
	当待排序的元素的值都来自于一个已知的小集合,如果元素的值是位于下界l和上界u之间的整数,
	我们可以计算每个这样的值出现的概率,然后把它们存储在数组F中。然后把有序列表的前F[0]个位置填入l,
	接下来的F[1]个位置填入l+1,以此类推。

\subsubsection{算法代码实现}
	\begin{lstlisting}
	// A中元素的值是位于下界l和上界u之间的整数
	// 分布计算相当于桶排序
	void distribution_counting(int *A, int len)
	{
		int i,j;
		int max = A[0], min = A[0];
		int* bucket = (int*)malloc((max - min + 1) * sizeof(int));
		int* result = (int*)malloc(len * sizeof(int));
		for(i = 0; i < len; ++i)
		{
			if(max < A[i])
				max = A[i];
			if(min > A[i])
				min = A[i];
		}
		for(i = 0; i < max-min+1; ++i)
			bucket[i] = 0;
		for(i = 0; i < len; ++i)
			bucket[A[i] - min] = bucket[A[i] - min] + 1;
		for(i = 1; i < max-min+1; ++i)
			bucket[i] = bucket[i-1] + bucket[i];
		for(i = len-1; i >= 0; --i)
		{
			j = A[i] - min;
			result[bucket[j] - 1] = A[i];
			--bucket[j];
		}
		free(bucket);
		free(result);
	}
	\end{lstlisting}

\section{串匹配中的输入增强技术}
\section{散列法}
\section{B树}

\end{document}
