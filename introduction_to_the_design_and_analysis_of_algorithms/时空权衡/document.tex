% !TeX spellcheck = en_US
%% 字体:方正静蕾简体
%%		 方正粗宋
\documentclass[a4paper,left=2.5cm,right=2.5cm,11pt]{article}

\usepackage[utf8]{inputenc}
\usepackage{fontspec}
\usepackage{cite}
\usepackage{xeCJK}
\usepackage{indentfirst}
\usepackage{titlesec}
\usepackage{longtable}
\usepackage{graphicx}
\usepackage{float}
\usepackage{rotating}
\usepackage{subfigure}
\usepackage{tabu}
\usepackage{amsmath}
\usepackage{setspace}
\usepackage{amsfonts}
\usepackage{appendix}
\usepackage{listings}
\usepackage{xcolor}
\usepackage{geometry}
\setcounter{secnumdepth}{4}
\usepackage{mhchem}
\usepackage{multirow}
\usepackage{extarrows}
\usepackage{hyperref}
\titleformat*{\section}{\LARGE}
\renewcommand\refname{参考文献}
\renewcommand{\abstractname}{\sihao \cjkfzcs 摘{  }要}
%\titleformat{\chapter}{\centering\bfseries\huge\wryh}{}{0.7em}{}{}
%\titleformat{\section}{\LARGE\bf}{\thesection}{1em}{}{}
\titleformat{\subsection}{\Large\bfseries}{\thesubsection}{1em}{}{}
\titleformat{\subsubsection}{\large\bfseries}{\thesubsubsection}{1em}{}{}
\renewcommand{\contentsname}{{\cjkfzcs \centerline{目{  } 录}}}
\setCJKfamilyfont{cjkhwxk}{STXingkai}
\setCJKfamilyfont{cjkfzcs}{STSongti-SC-Regular}
% \setCJKfamilyfont{cjkhwxk}{华文行楷}
% \setCJKfamilyfont{cjkfzcs}{方正粗宋简体}
\newcommand*{\cjkfzcs}{\CJKfamily{cjkfzcs}}
\newcommand*{\cjkhwxk}{\CJKfamily{cjkhwxk}}
\newfontfamily\wryh{Microsoft YaHei}
\newfontfamily\hwzs{STZhongsong}
\newfontfamily\hwst{STSong}
\newfontfamily\hwfs{STFangsong}
\newfontfamily\jljt{MicrosoftYaHei}
\newfontfamily\hwxk{STXingkai}
% \newfontfamily\hwzs{华文中宋}
% \newfontfamily\hwst{华文宋体}
% \newfontfamily\hwfs{华文仿宋}
% \newfontfamily\jljt{方正静蕾简体}
% \newfontfamily\hwxk{华文行楷}
\newcommand{\verylarge}{\fontsize{60pt}{\baselineskip}\selectfont}  
\newcommand{\chuhao}{\fontsize{44.9pt}{\baselineskip}\selectfont}  
\newcommand{\xiaochu}{\fontsize{38.5pt}{\baselineskip}\selectfont}  
\newcommand{\yihao}{\fontsize{27.8pt}{\baselineskip}\selectfont}  
\newcommand{\xiaoyi}{\fontsize{25.7pt}{\baselineskip}\selectfont}  
\newcommand{\erhao}{\fontsize{23.5pt}{\baselineskip}\selectfont}  
\newcommand{\xiaoerhao}{\fontsize{19.3pt}{\baselineskip}\selectfont} 
\newcommand{\sihao}{\fontsize{14pt}{\baselineskip}\selectfont}      % 字号设置  
\newcommand{\xiaosihao}{\fontsize{12pt}{\baselineskip}\selectfont}  % 字号设置  
\newcommand{\wuhao}{\fontsize{10.5pt}{\baselineskip}\selectfont}    % 字号设置  
\newcommand{\xiaowuhao}{\fontsize{9pt}{\baselineskip}\selectfont}   % 字号设置  
\newcommand{\liuhao}{\fontsize{7.875pt}{\baselineskip}\selectfont}  % 字号设置  
\newcommand{\qihao}{\fontsize{5.25pt}{\baselineskip}\selectfont}    % 字号设置 

\usepackage{diagbox}
\usepackage{multirow}
\boldmath
\XeTeXlinebreaklocale "zh"
\XeTeXlinebreakskip = 0pt plus 1pt minus 0.1pt
\definecolor{cred}{rgb}{0.8,0.8,0.8}
\definecolor{cgreen}{rgb}{0,0.3,0}
\definecolor{cpurple}{rgb}{0.5,0,0.35}
\definecolor{cdocblue}{rgb}{0,0,0.3}
\definecolor{cdark}{rgb}{0.95,1.0,1.0}
\lstset{
	language=C,
	numbers=left,
	numberstyle=\tiny\color{black},
	showspaces=false,
	showstringspaces=false,
	basicstyle=\scriptsize,
	keywordstyle=\color{purple},
	commentstyle=\itshape\color{cgreen},
	stringstyle=\color{blue},
	frame=lines,
	% escapeinside=``,
	extendedchars=true, 
	xleftmargin=1em,
	xrightmargin=1em, 
	backgroundcolor=\color{cred},
	aboveskip=1em,
	breaklines=true,
	tabsize=4
} 

\newfontfamily{\consolas}{Consolas}
\newfontfamily{\monaco}{Monaco}
\setmonofont[Mapping={}]{Consolas}	%英文引号之类的正常显示,相当于设置英文字体
\setsansfont{Consolas} %设置英文字体 Monaco, Consolas,  Fantasque Sans Mono
\setmainfont{Times New Roman}

\setCJKmainfont{华文中宋}


\newcommand{\fic}[1]{\begin{figure}[H]
		\center
		\includegraphics[width=0.8\textwidth]{#1}
	\end{figure}}
	
\newcommand{\sizedfic}[2]{\begin{figure}[H]
		\center
		\includegraphics[width=#1\textwidth]{#2}
	\end{figure}}

\newcommand{\codefile}[1]{\lstinputlisting{#1}}

% 改变段间隔
\setlength{\parskip}{0.2em}
\linespread{1.1}

\usepackage{lastpage}
\usepackage{fancyhdr}
\pagestyle{fancy}
\lhead{\space \qquad \space}
\chead{第五章 减治法 \qquad}
\rhead{\qquad\thepage/\pageref{LastPage}}
\begin{document}

\tableofcontents

\clearpage

\section{插入排序}
\subsection{算法思路}
	为A[n-1]找到一个合适的位置,然后把它插入到那里。
	从右到左扫描这个有序的子数组,直到遇到第一个小于等于A[n-1],然后就把A[n-1]插在该元素的后面。
\subsection{算法实现代码}
	\begin{lstlisting}
	void insertion_sort(int *A, int len)
	{
		int i,j;
		int temp;
		for(i = 1; i < len; ++i)
		{
			temp = A[i];
			for(j = i - 1; j > 0; --j)
			{
				if(A[j] <= temp)
					break;
				A[j+1] = A[j];
			}
			A[j+1] = temp;
		}
	}
	\end{lstlisting}
\subsection{算法分析}
	\begin{itemize}
		\item 最坏情况下,输入是一个严格递减的数组,此时键值比较次数为
		\begin{equation}
			C_{worst}(n) = \frac{(n-1)n}{2} \in \Theta(n^2)
		\end{equation}
		\item 最坏情况下,插入排序和选择排序的键值比较次数是完全一致的。
		\item 最优情况下,输入是一个升序排列的数组。对于有序的数组,键值比较次数为
		\begin{equation}
			C_{best}(n) = n-1 \in \Theta(n)
		\end{equation}
		\item 该算法的平均键值比较次数是最差情况的一半,键值比较次数为
		\begin{equation}
			C_{avg}(n) \approx \frac{n^2}{4} \in \Theta(n^2)
		\end{equation}
	\end{itemize}

\section{深度优先查找}
\subsection{算法思路}
	从任意顶点开始访问图,然后将该顶点标记为已访问。
	在每次迭代的时候,该算法紧接着处理与当前顶点邻接的未访问顶点。
	这个过程一直持续,直到遇到一个无法通向未访问邻接顶点的顶点。
	此时,该算法沿着来路后退一条边,并试着从那里访问未访问的顶点。
	当后退到起始顶点的时候,如果起始顶点所邻接的所有顶点都被访问过了,那就从未访问过的顶点开始,重复上述过程。
\subsection{算法实现代码}
	\begin{lstlisting}
	typedef struct Vertex Vertex;
	struct Vertex
	{
		int order;
		int visited;
		int parent;
		int count;
		Vertex** edges;
	};
	void dfs(Vertex *v, int *count)
	{
		Vertex *edge;
		++(*count);
		v->visited = 1;
		visited->count = *count;
		for(edge = v->edges; edge != NULL; ++edge)
		{
			if(edge->visited == 0)
			{
				edge->parent = v->order;
				dfs(edge, *count);
			}
		}
	}
	void dfs_search(Vertex *V, int num)
	{
		int i;
		int count = 0;
		for(i = 0; i < num; ++i)
		{
			if(v[i].visited == 0)
			{
				(v+i)->parent = -1;
				dfs(v+i, &count);
			}
		}
	}
	\end{lstlisting}
\subsection{算法分析}
	\begin{itemize}
		\item 对于邻接矩阵表示法,该算法的时间复杂度为$\Theta(|V|^2)$。
		对于邻接链表表示法,该遍历的时间复杂度为$\Theta(|V|+|E|)$。
		\item 深度优先搜索可以检查图的连通性。当算法停止后,检查一下是否所有的顶点都被访问过了。
		如果都被访问过了,那么这个图是连通的,否则它是不连通的。
		\item 深度优先搜素可以检查图的无环性。利用DFS森林形式的表示法,如果DFS森林不包含回边,那么这个图就是无环的。
	\end{itemize}

\section{广度优先查找}
\subsection{算法分析}
	首先访问所有和初始顶点邻接的顶点,然后是离它两条边的所有未访问顶点。
	以此类推,直到所有与初始化顶点同在一个连通分量中的顶点都访问过了为止。
	如果仍然存在未被访问的顶点,该算法必须从该顶点,重复上述过程。

\subsection{算法实现代码}
	\begin{lstlisting}
	typedef struct Vertex Vertex;
	struct Vertex
	{
		int visited;
		int count;
		Vertex** edges;
		int parent;
		int order;
	};

	void bfs(Vertex* v, int* count, int num)
	{
		++(*count);
		v->count = *count;
		v->visited = 1;
		int i;
		int front = 0;
		int rear = -1;
		Vertex* temp;
		Vertex* adjacent;
		Vertex** queue = (Vertex**)malloc(num * sizeof(Vertex*));
		queue[++rear] = v;
		while(front <= rear)
		{
			temp = queue[front++];
			for(i = 0; temp->edges[i] != NULL; ++i)
			{
				adjacent = temp->edges[i];
				if(adjacent->visited == 0)
				{
					++(*count);
					adjacent->visited = 1;
					adjacent->parent = temp->order;
					adjacent->count = *count;
					queue[++rear] = adjacent;
				}
			}
		}
		free(queue);
	}

	void bfs_search(Vertex* v, int num)
	{
		int count = 0;
		int i;
		for(i = 0; i < num; ++i)
		{
			if(visited[i].visited == 0)
			{
				v->parent = -1;
				bfs(v, &count, num);
			}
		}
	}
	\end{lstlisting}

\subsection{算法分析}
	\begin{itemize}
		\item 对于邻接矩阵表示法,广度优先搜索的时间复杂度属于$\Theta(|V|^2)$。
		而对于邻接链表表示法,它属于$\Theta(|V|+|E|)$。
		\item BFS森林有两种不同类型的边:树向边和交叉边。树向边是指向原先未访问顶点的边,交叉边是指向那些已访问顶点的边。
		\item 类似于DFS的做法,可以用BFS来检查图的连通性和无环性。
		\item BFS可以用来求两个给定顶点间边数量最少的路径。 
	\end{itemize}

\section{拓扑排序}
\section{生成排列}
\section{生成子集}
\section{假币问题}
\section{俄式乘法}
\section{约瑟夫斯问题}
\section{计算中值和选择问题}
\section{插值查找}
\section{二叉查找树的查找和插入}

\end{document}
