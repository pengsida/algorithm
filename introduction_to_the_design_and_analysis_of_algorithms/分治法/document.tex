% !TeX spellcheck = en_US
%% 字体:方正静蕾简体
%%		 方正粗宋
\documentclass[a4paper,left=2.5cm,right=2.5cm,11pt]{article}

\usepackage[utf8]{inputenc}
\usepackage{fontspec}
\usepackage{cite}
\usepackage{xeCJK}
\usepackage{indentfirst}
\usepackage{titlesec}
\usepackage{longtable}
\usepackage{graphicx}
\usepackage{float}
\usepackage{rotating}
\usepackage{subfigure}
\usepackage{tabu}
\usepackage{amsmath}
\usepackage{setspace}
\usepackage{amsfonts}
\usepackage{appendix}
\usepackage{listings}
\usepackage{xcolor}
\usepackage{geometry}
\setcounter{secnumdepth}{4}
\usepackage{mhchem}
\usepackage{multirow}
\usepackage{extarrows}
\usepackage{hyperref}
\titleformat*{\section}{\LARGE}
\renewcommand\refname{参考文献}
\renewcommand{\abstractname}{\sihao \cjkfzcs 摘{  }要}
%\titleformat{\chapter}{\centering\bfseries\huge\wryh}{}{0.7em}{}{}
%\titleformat{\section}{\LARGE\bf}{\thesection}{1em}{}{}
\titleformat{\subsection}{\Large\bfseries}{\thesubsection}{1em}{}{}
\titleformat{\subsubsection}{\large\bfseries}{\thesubsubsection}{1em}{}{}
\renewcommand{\contentsname}{{\cjkfzcs \centerline{目{  } 录}}}
\setCJKfamilyfont{cjkhwxk}{STXingkai}
\setCJKfamilyfont{cjkfzcs}{STSongti-SC-Regular}
% \setCJKfamilyfont{cjkhwxk}{华文行楷}
% \setCJKfamilyfont{cjkfzcs}{方正粗宋简体}
\newcommand*{\cjkfzcs}{\CJKfamily{cjkfzcs}}
\newcommand*{\cjkhwxk}{\CJKfamily{cjkhwxk}}
\newfontfamily\wryh{Microsoft YaHei}
\newfontfamily\hwzs{STZhongsong}
\newfontfamily\hwst{STSong}
\newfontfamily\hwfs{STFangsong}
\newfontfamily\jljt{MicrosoftYaHei}
\newfontfamily\hwxk{STXingkai}
% \newfontfamily\hwzs{华文中宋}
% \newfontfamily\hwst{华文宋体}
% \newfontfamily\hwfs{华文仿宋}
% \newfontfamily\jljt{方正静蕾简体}
% \newfontfamily\hwxk{华文行楷}
\newcommand{\verylarge}{\fontsize{60pt}{\baselineskip}\selectfont}  
\newcommand{\chuhao}{\fontsize{44.9pt}{\baselineskip}\selectfont}  
\newcommand{\xiaochu}{\fontsize{38.5pt}{\baselineskip}\selectfont}  
\newcommand{\yihao}{\fontsize{27.8pt}{\baselineskip}\selectfont}  
\newcommand{\xiaoyi}{\fontsize{25.7pt}{\baselineskip}\selectfont}  
\newcommand{\erhao}{\fontsize{23.5pt}{\baselineskip}\selectfont}  
\newcommand{\xiaoerhao}{\fontsize{19.3pt}{\baselineskip}\selectfont} 
\newcommand{\sihao}{\fontsize{14pt}{\baselineskip}\selectfont}      % 字号设置  
\newcommand{\xiaosihao}{\fontsize{12pt}{\baselineskip}\selectfont}  % 字号设置  
\newcommand{\wuhao}{\fontsize{10.5pt}{\baselineskip}\selectfont}    % 字号设置  
\newcommand{\xiaowuhao}{\fontsize{9pt}{\baselineskip}\selectfont}   % 字号设置  
\newcommand{\liuhao}{\fontsize{7.875pt}{\baselineskip}\selectfont}  % 字号设置  
\newcommand{\qihao}{\fontsize{5.25pt}{\baselineskip}\selectfont}    % 字号设置 

\usepackage{diagbox}
\usepackage{multirow}
\boldmath
\XeTeXlinebreaklocale "zh"
\XeTeXlinebreakskip = 0pt plus 1pt minus 0.1pt
\definecolor{cred}{rgb}{0.8,0.8,0.8}
\definecolor{cgreen}{rgb}{0,0.3,0}
\definecolor{cpurple}{rgb}{0.5,0,0.35}
\definecolor{cdocblue}{rgb}{0,0,0.3}
\definecolor{cdark}{rgb}{0.95,1.0,1.0}
\lstset{
	language=C,
	numbers=left,
	numberstyle=\tiny\color{black},
	showspaces=false,
	showstringspaces=false,
	basicstyle=\scriptsize,
	keywordstyle=\color{purple},
	commentstyle=\itshape\color{cgreen},
	stringstyle=\color{blue},
	frame=lines,
	% escapeinside=``,
	extendedchars=true, 
	xleftmargin=1em,
	xrightmargin=1em, 
	backgroundcolor=\color{cred},
	aboveskip=1em,
	breaklines=true,
	tabsize=4
} 

\newfontfamily{\consolas}{Consolas}
\newfontfamily{\monaco}{Monaco}
\setmonofont[Mapping={}]{Consolas}	%英文引号之类的正常显示,相当于设置英文字体
\setsansfont{Consolas} %设置英文字体 Monaco, Consolas,  Fantasque Sans Mono
\setmainfont{Times New Roman}

\setCJKmainfont{华文中宋}


\newcommand{\fic}[1]{\begin{figure}[H]
		\center
		\includegraphics[width=0.8\textwidth]{#1}
	\end{figure}}
	
\newcommand{\sizedfic}[2]{\begin{figure}[H]
		\center
		\includegraphics[width=#1\textwidth]{#2}
	\end{figure}}

\newcommand{\codefile}[1]{\lstinputlisting{#1}}

% 改变段间隔
\setlength{\parskip}{0.2em}
\linespread{1.1}

\usepackage{lastpage}
\usepackage{fancyhdr}
\pagestyle{fancy}
\lhead{\space \qquad \space}
\chead{第四章 分治法 \qquad}
\rhead{\qquad\thepage/\pageref{LastPage}}
\begin{document}

\tableofcontents

\clearpage

\section{主定理}
	分治法的运用:将一个规模为n的实例分为若干个规模为n/b的实例,其中a个实例需要求解。
	对于算法的运行时间,有如下公式:
	\begin{equation}
		T(n) = aT(n/b) + f(n)
	\end{equation}

	其中,f(n)是一个函数,用于表示将问题分解为小问题和将结果合并起来所消耗的时间。\par

	可以使用主定理计算算法复杂度:
	\begin{equation}
	T(n) \in
		\begin{cases}
			\Theta(n^d), &\text{当}a<b^d\text{时}\\
			\Theta(n^d \log n), &\text{当}a=b^d\text{时}\\
			\Theta(n^{\log_d^a}), &\text{当}a>b^d\text{时}\\
		\end{cases}
	\end{equation}

\section{合并排序}
	算法思路:对于一个需要排序的数组A[0...n-1],将其分为$A[0, \lfloor n/2 \rfloor-1]$和$A[\lfloor n/2 \rfloor ... n-1]$,
	并对每个字数组递归排序,然后把这两个排好序的子数组合并为一个有序数组。\par

	算法实现如下:
	\begin{lstlisting}
	void merge(int *B, int *C, int *A, int len_b, int len_c, int left)
	{
		int index_b = 0;
		int index_c = 0;
		while(index_b < len_b && index_c < len_c)
		{
			if(B[index_b] < C[index_c])
				A[left++] = B[index_b++];
			else
				A[left++] = C[index_c++];
		}
		while(index_b < len_b)
			A[left++] = B[index_b++];
		while(index_c < len_c)
			A[left++] = C[index_c++];
	}

	// A的元素范围为[left, right)
	void merge_sort(int *A, int left, int right)
	{
		int i;
		int j = 0;
		int len = right - left;
		if(len > 1)
		{
			int mid = (right + left) / 2;
			int *B = (int*)malloc((len/2)*sizeof(int));
			int *C = (int*)malloc((len - len/2)*sizeof(int));

			for(i = 0; i < len/2; ++i)
				B[i] = A[i];
			for(; i < len; ++i)
				C[j++] = A[i];

			merge_sort(B, left, mid);
			merge_sort(C, mid, right);
			merge(B, C, A, len/2, len-len/2, left);
			free(B);
			free(C);
		}
	}
	\end{lstlisting}

	算法分析:
	\begin{itemize}
		\item 算法基本操作是键值比较操作,键值比较次数$C(n)$的递推关系式如下:
		\begin{equation}
			\text{当}n>1\text{时},C(n) = 2C(n/2) + C_{merge}(n),C(1) = 0
		\end{equation}
		
		\item $C_{merge}(n)$是合并阶段进行键值比较的次数。在最坏情况下,$C_{merge}(n)=n-1$。
		\item 在最坏情况下,如果$n=2^k$,最差效率递推式的精确解为$C_{worst}=n\log_2^n-n+1$。
		\item 在最坏情况下,算法时间复杂度为$C_{worst}(n) \in \Theta(n\log^n)$。
		\item 合并排序的主要缺点就是该算法需要线性的额外空间。
	\end{itemize}

\section{快速排序}
	算法思路:对给定数组中的元素进行重新排列,得到一个分区。在这个分区中,所有在s下标之前的元素都小等于A[s],
	所有在s下标之后的元素都大等于A[s]。随后对A[s]前和A[s]后的子数组进行相同的操作。\par

	算法实现如下:
	\begin{lstlisting}
	int partition(int *A, int left, int right)
	{
		int index = left;
		int pivot = A[right];
		int i;
		int temp;
		for(i = left; i < right -1; ++i)
		{
			if(A[i] < pivot)
			{
				temp = A[index];
				A[index] = A[i];
				A[i] = temp;
				++index;
			}
		}
		temp = A[index];
		A[index] = A[right];
		A[right] = temp;
		return index;
	}
	int hoare_partition(int *A, int left, int right)
	{
		int index = left;
		int pivot = A[left++];
		int temp;
		while(left < right)
		{
			while(A[left] < pivot)
				++left;
			while(A[right] > pivot)
				--right;
			if(left >= right)
				break;
			temp = A[right];
			A[right] = A[left];
			A[left] = temp;			
		}
		temp = A[right];
		A[right] = A[index];
		A[index] = temp;
		return right;
	}
	// A的元素范围为[left, right]
	void quick_sort(int *A, int left, int right)
	{
		int mid;
		if(left < right)
		{
			mid = partition(A, left, right);
			quick_sort(A, left, mid-1);
			quick_sort(A, mid+1, right);
		}
	}
	\end{lstlisting}

	算法分析:
	\begin{itemize}
		\item 如果扫描指针交叉,那么建立分区之前所执行的键值比较次数是n+1;如果它们相等,那么键值比较次数是n。
		\item 最优情况是所有分裂点位于相应子数组的中点。在最优情况下,键值比较次数$C_{best}(n)$满足下面的递推式:
		\begin{equation}
			\text{当n>1时},C_{best}(n) = 2C_{best}(n/2)+n,C_{best}(1)=0
		\end{equation}

		根据主定理,$C_{best}(n) \in \Theta(n\log_2^n)$。当$n=2^k$时,$C_{best}(n)=n\log_2^n$。
		\item 在最差的情况下,所有的分裂点抖趋于极端:两个子数组有一个为空,而另一个子数组仅仅比被分区的数组少一个元素。
		当输入的数组已经被排过序时,最差情况就会发生。最差情况下键值比较次数为$C_{worst}(n)=\frac{(n+1)(n+2)}{2}-3 \in \Theta(n^2)$。
		\item 平均键值比较次数$C_{avg}(n) \approx 2n\ln{n} \approx 1.38n\log_2^n$。
	\end{itemize}

\section{折半查找}
	算法思路:对于一列有序数组,比较查找键K和数组中间元素A[m]。如果它们相等,则算法结束。如果K<A[m],则对数组的前半部分执行该操作。
	如果K>A[m],则对数组的后半部分执行该操作。\par
	算法实现如下:
	\begin{lstlisting}
	int binary_search(int *A, int K, int len)
	{
		int left = 0;
		int right = len-1;
		int mid;
		while(left <= right)
		{
			mid = (left + right) / 2;
			if(K == A[mid])
				return mid;
			else if(K < A[mid])
				right = mid - 1;
			else
				left = mid + 1;
		}
		return -1;
	}
	\end{lstlisting}

	算法分析:
	\begin{itemize}
		\item 键值比较次数不仅取决于n,还取决于输入的特征。
		\item  
		在最坏的情况下,算法在进行了一次比较后,除了数组规模变为原来的二分之一,算法仍然面临同样的情况。
		最坏情况下的键值比较次数有如下递推式:
		\begin{equation}
			\text{当n>1时}, C_w(n) = C_w(\lfloor n/2 \rfloor) + 1, C_w(1) = 1
		\end{equation}

		对于任意的正整数n,$C_w(n) = \lfloor \log_2^n \rfloor + 1 = \lceil \log_2(n+1) \rceil$。
		\item 在最差情况下,折半查找的时间复杂度为$\log_2^n$。
		\item 折半查找的平均键值比较次数为$C_{avg}(n) \approx \log_2^n$。在查找成功的情况下,$C_{avg}(n) \approx \log_2(n-1)$。
		在查找失败的情况下,$C_{avg}(n) \approx \log_2(n+1)$。
	\end{itemize}

\section{二叉树的遍历}
	二叉树高度的定义:二叉树高度是根的左、右子树的最大高度加一。空树的高度为-1。\par

	计算二叉树高度的代码如下:
	\begin{lstlisting}
	typedef struct tree_node tree_node;
	struct tree_node
	{
		int key;
		tree_node* left;
		tree_node* right;
	}
	int max(int a, int b)
	{
		if(a > b)
			return a;
		return b;
	}
	int height(tree_node* root)
	{
		if(root == NULL)
			return -1;
		return max(height(root->left), height(root->right)) + 1;
	}
	\end{lstlisting}

	算法分析:
	\begin{itemize}
		\item 算法的基本操作是检查树是否为空。
		\item 外部顶点的数量总是比内部顶点的数量大一。
		\item 检查树是否为空的比较操作次数为$C(n) = 2n + 1$,加法操作的次数为$A(n) = n$。
	\end{itemize}

	二叉树的三种经典遍历算法如下:
	\begin{itemize}
		\item 在前序遍历中,根在访问左右子树之前被访问。
		\item 在中序遍历中,根在访问左子树之后,但在访问右子树之前被访问。
		\item 在后序遍历中,根在访问左右子树之后被访问。
	\end{itemize}

\section{大整数乘法}
\section{Strassen矩阵乘法}
\section{用分治法解最近对问题}
\section{用分治法解凸包问题}

\end{document}
