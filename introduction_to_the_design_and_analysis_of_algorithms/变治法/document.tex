% !TeX spellcheck = en_US
%% 字体:方正静蕾简体
%%		 方正粗宋
\documentclass[a4paper,left=2.5cm,right=2.5cm,11pt]{article}

\usepackage[utf8]{inputenc}
\usepackage{fontspec}
\usepackage{cite}
\usepackage{xeCJK}
\usepackage{indentfirst}
\usepackage{titlesec}
\usepackage{longtable}
\usepackage{graphicx}
\usepackage{float}
\usepackage{rotating}
\usepackage{subfigure}
\usepackage{tabu}
\usepackage{amsmath}
\usepackage{setspace}
\usepackage{amsfonts}
\usepackage{appendix}
\usepackage{listings}
\usepackage{xcolor}
\usepackage{geometry}
\setcounter{secnumdepth}{4}
\usepackage{mhchem}
\usepackage{multirow}
\usepackage{extarrows}
\usepackage{hyperref}
\titleformat*{\section}{\LARGE}
\renewcommand\refname{参考文献}
\renewcommand{\abstractname}{\sihao \cjkfzcs 摘{  }要}
%\titleformat{\chapter}{\centering\bfseries\huge\wryh}{}{0.7em}{}{}
%\titleformat{\section}{\LARGE\bf}{\thesection}{1em}{}{}
\titleformat{\subsection}{\Large\bfseries}{\thesubsection}{1em}{}{}
\titleformat{\subsubsection}{\large\bfseries}{\thesubsubsection}{1em}{}{}
\renewcommand{\contentsname}{{\cjkfzcs \centerline{目{  } 录}}}
\setCJKfamilyfont{cjkhwxk}{STXingkai}
\setCJKfamilyfont{cjkfzcs}{STSongti-SC-Regular}
% \setCJKfamilyfont{cjkhwxk}{华文行楷}
% \setCJKfamilyfont{cjkfzcs}{方正粗宋简体}
\newcommand*{\cjkfzcs}{\CJKfamily{cjkfzcs}}
\newcommand*{\cjkhwxk}{\CJKfamily{cjkhwxk}}
\newfontfamily\wryh{Microsoft YaHei}
\newfontfamily\hwzs{STZhongsong}
\newfontfamily\hwst{STSong}
\newfontfamily\hwfs{STFangsong}
\newfontfamily\jljt{MicrosoftYaHei}
\newfontfamily\hwxk{STXingkai}
% \newfontfamily\hwzs{华文中宋}
% \newfontfamily\hwst{华文宋体}
% \newfontfamily\hwfs{华文仿宋}
% \newfontfamily\jljt{方正静蕾简体}
% \newfontfamily\hwxk{华文行楷}
\newcommand{\verylarge}{\fontsize{60pt}{\baselineskip}\selectfont}  
\newcommand{\chuhao}{\fontsize{44.9pt}{\baselineskip}\selectfont}  
\newcommand{\xiaochu}{\fontsize{38.5pt}{\baselineskip}\selectfont}  
\newcommand{\yihao}{\fontsize{27.8pt}{\baselineskip}\selectfont}  
\newcommand{\xiaoyi}{\fontsize{25.7pt}{\baselineskip}\selectfont}  
\newcommand{\erhao}{\fontsize{23.5pt}{\baselineskip}\selectfont}  
\newcommand{\xiaoerhao}{\fontsize{19.3pt}{\baselineskip}\selectfont} 
\newcommand{\sihao}{\fontsize{14pt}{\baselineskip}\selectfont}      % 字号设置  
\newcommand{\xiaosihao}{\fontsize{12pt}{\baselineskip}\selectfont}  % 字号设置  
\newcommand{\wuhao}{\fontsize{10.5pt}{\baselineskip}\selectfont}    % 字号设置  
\newcommand{\xiaowuhao}{\fontsize{9pt}{\baselineskip}\selectfont}   % 字号设置  
\newcommand{\liuhao}{\fontsize{7.875pt}{\baselineskip}\selectfont}  % 字号设置  
\newcommand{\qihao}{\fontsize{5.25pt}{\baselineskip}\selectfont}    % 字号设置 

\usepackage{diagbox}
\usepackage{multirow}
\boldmath
\XeTeXlinebreaklocale "zh"
\XeTeXlinebreakskip = 0pt plus 1pt minus 0.1pt
\definecolor{cred}{rgb}{0.8,0.8,0.8}
\definecolor{cgreen}{rgb}{0,0.3,0}
\definecolor{cpurple}{rgb}{0.5,0,0.35}
\definecolor{cdocblue}{rgb}{0,0,0.3}
\definecolor{cdark}{rgb}{0.95,1.0,1.0}
\lstset{
	language=C,
	numbers=left,
	numberstyle=\tiny\color{black},
	showspaces=false,
	showstringspaces=false,
	basicstyle=\scriptsize,
	keywordstyle=\color{purple},
	commentstyle=\itshape\color{cgreen},
	stringstyle=\color{blue},
	frame=lines,
	% escapeinside=``,
	extendedchars=true, 
	xleftmargin=1em,
	xrightmargin=1em, 
	backgroundcolor=\color{cred},
	aboveskip=1em,
	breaklines=true,
	tabsize=4
} 

\newfontfamily{\consolas}{Consolas}
\newfontfamily{\monaco}{Monaco}
\setmonofont[Mapping={}]{Consolas}	%英文引号之类的正常显示,相当于设置英文字体
\setsansfont{Consolas} %设置英文字体 Monaco, Consolas,  Fantasque Sans Mono
\setmainfont{Times New Roman}

\setCJKmainfont{华文中宋}


\newcommand{\fic}[1]{\begin{figure}[H]
		\center
		\includegraphics[width=0.8\textwidth]{#1}
	\end{figure}}
	
\newcommand{\sizedfic}[2]{\begin{figure}[H]
		\center
		\includegraphics[width=#1\textwidth]{#2}
	\end{figure}}

\newcommand{\codefile}[1]{\lstinputlisting{#1}}

% 改变段间隔
\setlength{\parskip}{0.2em}
\linespread{1.1}

\usepackage{lastpage}
\usepackage{fancyhdr}
\pagestyle{fancy}
\lhead{\space \qquad \space}
\chead{第六章 变治法 \qquad}
\rhead{\qquad\thepage/\pageref{LastPage}}
\begin{document}

\tableofcontents

\clearpage

\section{预排序}
\subsection{检验数组中元素的唯一性}
	蛮力算法:对数组中的元素对进行比较,直到找到了两个相等的元素,或者所有的元素对都已比较完毕。它的最差效率属于$\Theta(n^2)$。\par
	变治法的思想:先对数组排序,然后只检查它的连续元素。如果该数组有相等的元素,则一定有一对元素是相互紧挨着的。\par
	实现代码如下:
	\begin{lstlisting}
	bool presort_element_uniqueness(int *A, int len)
	{
		int i;
		quick_sort(A, len);
		for(i = 0; i < len-1; ++i)
		{
			if(A[i] == A[i+1])
				return false;
		}
		return true;
	}
	\end{lstlisting}

	算法分析:算法总时间为用于排序的时间加上用于检验连续元素的时间。前者至少有$n\log n$次比较,而后者比较次数不会超过$n-1$。

\subsection{模式计算}
	在给定的数字列表中最经常出现的一个数值称为模式。\par
	蛮力法的思想:在另一个列表中存储已经遇到的值和它们的出现频率。在每次迭代当中,通过遍历这个辅助列表,
	原始列表中的第i个元素要和已遇到的数值进行比较。如果遇到一个匹配数值,该数值的出现次数加一。
	否则将当前元素添加到辅助列表中,并把它的出现次数置为一。\par
	基于蛮力法的算法,它的最差输入是一个没有相等元素的列表。在最差情况下,该算法的比较次数为$C(n) = \frac{(n-1)n}{2} \in \Theta(n^2)$。\par
	变治法的思想:先对输入排序,然后求出在该有序数组中邻接次数最多的等值元素,相应地就求出了模式。

	\begin{lstlisting}
	int presort_mode(int *A, int len)
	{
		int i = 0;
		int mode = -1;
		int max_length = 0;
		int length = 0;
		int temp;
		quick_sort(A, len);
		while(i < len)
		{
			length = 1;
			temp = A[i];
			while(i < len-1 && temp == A[i+1])
			{
				++i;
				++length;
			}
			if(length > max_length)
			{
				mode = temp;
				max_length = length;
			}
			++i;
		}
		return mode;
	}
	\end{lstlisting}

	该算法的运行时间受限于排序时间。

\subsection{查找问题}
	蛮力法的思想:顺序查找,最差情况下需要进行n次比较。\par
	变治法的思想:预排序,然后应用折半查找。这个查找算法在最差情况下的总运行时间是$\Theta(n\log n)$。

\section{高斯消去法}
\subsection{算法实现代码}
	\begin{lstlisting}
	void gauss_elimination(double **A, double *b, int row, int col)
	{
		int i,j,k;
		double temp;
		for(i = 0; i < row; ++i)
		{
			for(j = i+1; j < row; ++j)
			{
				if(A[i][i] != 0)
					temp = A[j][i] / A[i][i];
				else
					temp = 0;
				for(k = i; k < col; ++k)
					A[j][k] = A[j][k] - temp * A[i][k];
				b[j] = b[j] - temp * b[i];
			}
		}
	}
	\end{lstlisting}

\subsection{算法改进}
	上述算法存在一个问题:$A[i][i]$可能会非常小,导致比例因子$A[j][i]/A[i][i]$非常大。\par
	改进:每次都去找第i列系数的绝对值最大的行,然后把它作为第i次迭代的基点。这种修改称为部分选主元法。\par
	改进算法的实现代码如下:
	\begin{lstlisting}
	void better_gauss_elimination(double **A, double *b, int row, int col)
	{
		int i,j,k;
		int max_row;
		double temp;
		for(i = 0; i < row; ++i)
		{
			max_row = i;
			for(j = i+1; j < row; ++j)
			{
				if(A[j][i] > A[max_row][i])
					max_row = j;
			}
			for(k = i; k < col; ++k)
			{
				temp = A[i][k];
				A[i][k] = A[max_row][k];
				A[max_row][k] = temp;
			}
			temp = b[i];
			b[i] = b[max_row];
			b[max_row] = temp;
			for(j = i+1; j < row; ++j)
			{
				temp = A[j][i] / A[i][i];
				for(k = i; k < col; ++k)
					A[j][k] = A[j][k] - temp * A[i][k];
				b[j] = b[j] - temp * b[i];
			} 
		}
	}
	\end{lstlisting}

\subsection{算法分析}
	算法的基本操作是乘法操作。乘法操作次数为
	\begin{equation}
		C(n) \approx \frac{1}{3}n^3 \in \Theta(n^3)
	\end{equation}

\subsection{LU分解}
	高斯消去法的现代商业实现以LU分解为基础。\par
	矩阵A可以分解为下三角矩阵L和上三角矩阵U。L矩阵由主对角线上的1和高斯消去过程中行的乘数所构成的。U矩阵是对A进行高斯消去后得到的结果。\par
	解方程组$Ax=b$就等价于解方程组$LUx=b$。
	先设$y=Ux$,那么$Ly=b$,就能求出y。然后求解方程组$Ux=y$,就能求出x。\par
	LU分解的优点:只要得到了矩阵A的LU分解,无论对于什么样的右边向量b,都可以利用矩阵L和矩阵U对其进行求解,不需要每次都进行高斯消去法。\par
	LU分解不需要额外的存储空间,我们可以把U的非0部分存储在A的上三角部分,把L的有效部分存储在A的主对角线的下方。

\subsection{计算矩阵的逆}
\subsection{计算矩阵的行列式}

\section{平衡查找树}
\section{堆和堆排序}
\section{霍纳法则}
\section{二进制幂}
\section{问题化简}

\end{document}
