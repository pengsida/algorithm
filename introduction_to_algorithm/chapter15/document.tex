% !TeX spellcheck = en_US
%% 字体:方正静蕾简体
%%		 方正粗宋
\documentclass[a4paper,left=2.5cm,right=2.5cm,11pt]{article}

\usepackage[utf8]{inputenc}
\usepackage{fontspec}
\usepackage{cite}
\usepackage{xeCJK}
\usepackage{indentfirst}
\usepackage{titlesec}
\usepackage{longtable}
\usepackage{graphicx}
\usepackage{float}
\usepackage{rotating}
\usepackage{subfigure}
\usepackage{tabu}
\usepackage{amsmath}
\usepackage{setspace}
\usepackage{amsfonts}
\usepackage{appendix}
\usepackage{listings}
\usepackage{xcolor}
\usepackage{geometry}
\setcounter{secnumdepth}{4}
\usepackage{mhchem}
\usepackage{multirow}
\usepackage{extarrows}
\usepackage{hyperref}
\titleformat*{\section}{\LARGE}
\renewcommand\refname{参考文献}
\renewcommand{\abstractname}{\sihao \cjkfzcs 摘{  }要}
%\titleformat{\chapter}{\centering\bfseries\huge\wryh}{}{0.7em}{}{}
%\titleformat{\section}{\LARGE\bf}{\thesection}{1em}{}{}
\titleformat{\subsection}{\Large\bfseries}{\thesubsection}{1em}{}{}
\titleformat{\subsubsection}{\large\bfseries}{\thesubsubsection}{1em}{}{}
\renewcommand{\contentsname}{{\cjkfzcs \centerline{目{  } 录}}}
\setCJKfamilyfont{cjkhwxk}{STXingkai}
\setCJKfamilyfont{cjkfzcs}{STSongti-SC-Regular}
% \setCJKfamilyfont{cjkhwxk}{华文行楷}
% \setCJKfamilyfont{cjkfzcs}{方正粗宋简体}
\newcommand*{\cjkfzcs}{\CJKfamily{cjkfzcs}}
\newcommand*{\cjkhwxk}{\CJKfamily{cjkhwxk}}
\newfontfamily\wryh{Microsoft YaHei}
\newfontfamily\hwzs{STZhongsong}
\newfontfamily\hwst{STSong}
\newfontfamily\hwfs{STFangsong}
\newfontfamily\jljt{MicrosoftYaHei}
\newfontfamily\hwxk{STXingkai}
% \newfontfamily\hwzs{华文中宋}
% \newfontfamily\hwst{华文宋体}
% \newfontfamily\hwfs{华文仿宋}
% \newfontfamily\jljt{方正静蕾简体}
% \newfontfamily\hwxk{华文行楷}
\newcommand{\verylarge}{\fontsize{60pt}{\baselineskip}\selectfont}  
\newcommand{\chuhao}{\fontsize{44.9pt}{\baselineskip}\selectfont}  
\newcommand{\xiaochu}{\fontsize{38.5pt}{\baselineskip}\selectfont}  
\newcommand{\yihao}{\fontsize{27.8pt}{\baselineskip}\selectfont}  
\newcommand{\xiaoyi}{\fontsize{25.7pt}{\baselineskip}\selectfont}  
\newcommand{\erhao}{\fontsize{23.5pt}{\baselineskip}\selectfont}  
\newcommand{\xiaoerhao}{\fontsize{19.3pt}{\baselineskip}\selectfont} 
\newcommand{\sihao}{\fontsize{14pt}{\baselineskip}\selectfont}      % 字号设置  
\newcommand{\xiaosihao}{\fontsize{12pt}{\baselineskip}\selectfont}  % 字号设置  
\newcommand{\wuhao}{\fontsize{10.5pt}{\baselineskip}\selectfont}    % 字号设置  
\newcommand{\xiaowuhao}{\fontsize{9pt}{\baselineskip}\selectfont}   % 字号设置  
\newcommand{\liuhao}{\fontsize{7.875pt}{\baselineskip}\selectfont}  % 字号设置  
\newcommand{\qihao}{\fontsize{5.25pt}{\baselineskip}\selectfont}    % 字号设置 

\usepackage{diagbox}
\usepackage{multirow}
\boldmath
\XeTeXlinebreaklocale "zh"
\XeTeXlinebreakskip = 0pt plus 1pt minus 0.1pt
\definecolor{cred}{rgb}{0.8,0.8,0.8}
\definecolor{cgreen}{rgb}{0,0.3,0}
\definecolor{cpurple}{rgb}{0.5,0,0.35}
\definecolor{cdocblue}{rgb}{0,0,0.3}
\definecolor{cdark}{rgb}{0.95,1.0,1.0}
\lstset{
	language=C,
	numbers=left,
	numberstyle=\tiny\color{black},
	showspaces=false,
	showstringspaces=false,
	basicstyle=\scriptsize,
	keywordstyle=\color{purple},
	commentstyle=\itshape\color{cgreen},
	stringstyle=\color{blue},
	frame=lines,
	% escapeinside=``,
	extendedchars=true, 
	xleftmargin=1em,
	xrightmargin=1em, 
	backgroundcolor=\color{cred},
	aboveskip=1em,
	breaklines=true,
	tabsize=4
} 

\newfontfamily{\consolas}{Consolas}
\newfontfamily{\monaco}{Monaco}
\setmonofont[Mapping={}]{Consolas}	%英文引号之类的正常显示,相当于设置英文字体
\setsansfont{Consolas} %设置英文字体 Monaco, Consolas,  Fantasque Sans Mono
\setmainfont{Times New Roman}

\setCJKmainfont{华文中宋}


\newcommand{\fic}[1]{\begin{figure}[H]
		\center
		\includegraphics[width=0.8\textwidth]{#1}
	\end{figure}}
	
\newcommand{\sizedfic}[2]{\begin{figure}[H]
		\center
		\includegraphics[width=#1\textwidth]{#2}
	\end{figure}}

\newcommand{\codefile}[1]{\lstinputlisting{#1}}

% 改变段间隔
\setlength{\parskip}{0.2em}

\setlength{\itemsep}{0pt}
\linespread{1.1}

\usepackage{lastpage}
\usepackage{fancyhdr}
\pagestyle{fancy}
\lhead{\space \qquad \space}
\chead{动态规划 \qquad}
\rhead{\qquad\thepage/\pageref{LastPage}}
\begin{document}

\section{动态规划的基本步骤}
\begin{itemize}
	\item 描述最优解的结构。
	\item 递归定义最优解的值。
	\item 按自底向上的方式计算最优解的值。
\end{itemize}

\section{装配线调度问题}
\subsection{问题描述}
	现在有两条装配线,编号为i,每一条装配线有n个装配站。编号为j,所以将装配线i的第j个装配站表示为$s_{i,j}$。每个装配站的时间都是不同的。我们把在装配站$s_{i,j}$上所需的装配时间记为$a_{i,j}$。零件进入装配线i所需要的时间记为$e_i$,离开装配线所需要的时间记为$x_i$。\par
	工厂经理可以将部分完成的零件从一条装配线移到另一条装配线上,把已经通过装配站$s_{i,j}$的零件从装配线i移动到另一个装配线的时间为$t_{i,j}$。\par
	现在的问题是,需要确定在装配线1和装配线2中选择哪些站,使得完成一个零件的时间最短。
\subsection{问题解决}
\subsubsection{描述最优解的结构}
	对于装配线调度问题,一个问题的最优解包含了子问题的一个最优解,这种性质称为最优子结构。\par
	我来解释一下这种性质。现在我们要找通过装配站$s_{i,j}$的最快路线。零件可能来自装配站$s_{1,j-1}$也可能来自装配站$s_{2,j-1}$。不管是来自哪个装配站,零件通过装配站$s_{1,j-1}$或$s_{2,j-1}$时,都需要保证它是经过最快路线的。这样一来,求解一个问题的最优解之前,我们可以先求出它的字问题的最优解。\par
	这样一来,我们就可以利用子问题的最优解来构造原问题的一个最优解。现在我们想求出通过$s_{1,j}$的最优路线:
	\begin{itemize}
		\item 求出装配站$s_{1,j-1}$的最快路线,加上$a_{1,j}$后得到总时间。
		\item 求出装配线$s_{2,j-1}$的最快路线,加上$t_{1,j-1}$和$a_{1,j}$得到总时间。
		\item 对比两种方案的总时间,选择时间较短的路线为最佳路线。
	\end{itemize}
\subsubsection{递归定义最优解的值}
	令$f_i[j]$表示一个零件从起点到装配站$s_{i,j}$的最快时间。令$f^*$表示完成一个零件所需要的最少时间。
	很容易知道有下列等式关系成立:
	\begin{equation}
		f^* = min(f_1[n]+x_1,f_2[n]+x_2)
	\end{equation}

	通过子问题求解原问题,我们容易有下列等式成立:
	\begin{equation}
	\begin{split}
		&f_1[j] = min(f_1[j-1]+a_{1,j},f_2[j-1]+t_{2,j-1}+a_{1,j})\\ 
		&f_2[j] = min(f_2[j-1]+a_{2,j},f_1[j-1]+t_{1,j-1}+a_{2,j})
	\end{split}
	\end{equation}

	而且问题的初始条件也是容易得到的:
	\begin{equation}
		\begin{split}
			& f_1[1] = e_1 + a_{1,1}\\
			& f_2[1] = e_2 + a_{2,1}
		\end{split}
	\end{equation}

\subsubsection{按自底向上的方式计算最优解的值}
	根据上一小节的公式,很容易就能编写出相应的伪代码。
	\begin{lstlisting}
		FASTEST_WAY(a,t,e,x,n)
			f1[1] = e1 + a1[1]
			f2[1] = e2 + a2[1]
			for j = 2 to n
				// 求f1[j]的最优路线
				// l1[j]用于存放上一个装配站
				if f1[j-1]+a1[j] <= f2[j-1]+t2[j-1]+a1[j]
					f1[j] = f1[j-1] + a1[j]
					l1[j] = 1
				else
					f1[j] = f2[j-1] + t2[j-1] + a1[j]
					l1[j] = 2
				// 求f2[j]的最优路线
				// l2[j]用于存放上一个装配站
				if f2[j-1]+a2[j] <= f1[j-1]+t1[j-1]+a2[j]
					f2[j] = f2[j-1] + a2[j]
					l2[j] = 2
				else
					f2[j] = f1[j-1] + t1[j-1] + a2[j]
					l2[j] = 1 
			if f1[n]+x1 <= f2[n]+x2
				f = f1[n] + x1
				l = 1
			else
				f = f2[n] + x2
				l = 2
			// 之后可以通过l、l1与l2中存放的路线构造出最快路线
	\end{lstlisting}
\end{document}
